\subsection{Ponds and Fields}
In this section, we describe the different techniques to choose ponds and fields and discuss their utility. We treat ponds and parks as the same (with respect to placement near a building) in our algorithm and analysis, since they generate the same increase in score. In every algorithm, we first check if the residence is already connected to an existing park or pond. If it is, we skip this step entirely.

\subsubsection{Utility}
We had doubts about whether ponds and parks were necessary at all in the beginning. In a previous section about analysing the runs, we added a metric, the utility of parks and ponds, that calculates the total points generated by ponds and parks, divided by the number of pond and park cells. This gives us the utility of a single pond or park cell. We discovered that in every single run, even a random walk generated pond or park had a utility of over 1.2. The increase in score between using no parks and ponds versus using parks and ponds was around 100 points, irrespective of the strategy we use, as long as there are a good number of residence buildings. The only other source of points is from buildings, which have a utility of only one point per cell, if you do not consider the roads that must be built to reach them. We concluded that it is always better to build parks and ponds.

\subsubsection{Number of Configurations}
First let us find a worst case of the number of possible ways to place a park or pond adjacent to a residence. The park or pond is at most four cells large. We found that using larger parks and ponds did not make sense, since we only need four to generate the score, and an extension to it could always be built later. These four cells are in the shape of a tetromino. There are a total of seven possible one sided (counting mirror reflections separately) tetrominos, each having a maximum perimeter of 10. The residence is a pentomino having a perimeter of size upto 12. The total number of ways to place the tetromino adjacent to the pentomino is 840. This is too large to run for each candidate building so we concluded that we had to run the pond and park generation algorithm after choosing a candidate location, rather than including it in our algorithm to choose the best move.

\subsubsection{Evaluating a Park or Pond}
We developed a metric to choose between two different park candidates. At first, we thought that like buildings, choosing the candidate with the largest perimeter would increase packing efficiency. However, the random walk discussion below showed that this metric actually decreased the score. We instead went with choosing the path that had the least perimeter instead, since this led to a larger score than maximising the perimeter. The reason for this is that placing a park or pond near buildings does not generate any points for them, since they have already been placed, thus wasting a large part of their perimeter.

\subsubsection{Random Walks}
The first approach to build parks and ponds was to simply generate a random walk of size four. However, this is clearly not a good strategy. We decided to instead generate a large number of random walks and choose the best one from them. The metrics we use to compare candidates for parks and ponds is given above. We tried generating 50, 100, 200 and 400 random walks. When we tried comparing them using the largest perimeter, the score actually decreased as we increased the number of candidates, leading us to believe that this metric was the opposite of what we needed. So we reversed it and chose the park or pond with the least perimeter. For 50, 100, 200 and 400 random walks, our score increased, with diminishing increases each time as expected. At 400 walks, our pond function started slowing down the program, while improving our score by only around five points (which is so low, it could even be experimental error), so we decided to stop at 200. 

\subsubsection{Connecting to Previous Parks and Ponds}
We already checked if the building was adjacent to a park or pond prior to constructing a new one. We added this feature to check if there was a park or pond within a distance of three cells away, so that we could instead extend it rather than build a new one. This is clearly better than building a new park or pond, because it uses fewer cells and generates the same score increase. We always choose to extend a previous park or pond if it is a possibility.

\subsubsection{Straight Parks and Ponds} 
We hypothesised that since long flat parks and ponds have a large perimeter, they would generate the highest score. Since we use a random walk, some of these might get missed, so we added a feature to include all possible configurations of straight parks and ponds for consideration by our metric, apart from the 200 randomly generated parks and ponds. This improved our score, but only slightly.




